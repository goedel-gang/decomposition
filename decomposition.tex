% Compiling with
% latexmk -halt-on-error -shell-escape -synctex=1 -pdf
% (Recommend using a latexmkrc file so as just to run latexmk -pvc, for example)
% Probably you can achieve the same with an inordinate number of invocations of
% pdflatex -halt-on-error -shell-escape -synctex=1

% fleqn aligns equations to the left, a4 paper size, 11pt font, article class
\documentclass[fleqn,a4paper,11pt]{article}
\title{On Decompositions of Infinite Groups}
\author{A. Horn, D. Simms, et al.}

\usepackage{mymaths}
\usepackage{mystyle}

\begin{document}
\maketitle

\section{\((\Integers, +)\)}
\(\Integers\) is not decomposable. Suppose we had
\(\Integers \isom G \times H\) for nontrivial \(G\), \(H\). Then since the
subgroups of \(\Integers\) are of the form \(k\Integers\), we have
\(G \isom n\Integers\) and \(H \isom m\Integers\) for some \(n, m\).

Since \(k\Integers \isom \Integers\), this implies
\(\Integers \isom G \times H \isom \Integers \times \Integers\). This is not
possible since \(\Integers\) has \(2\Integers\) as its only subgroup of index
two, but \(\Integers \times \Integers\) has two subgroups of index two:
\(2\Integers \times \Integers\) and \(\Integers \times 2\Integers\).

Hence \(\Integers\) is not decomposable.

\section{\((\Rationals, +)\)}

Suppose we had \(\Rationals \isom G \times H\) for some nontrivial groups
\(G, H\). Both \(G\) and \(H\) have a subgroup isomorphic to \(\Integers\),
since each contains a nonzero element, and each element of \(\Rationals\) has
infinite order.

Hence \(\Rationals\) has a subgroup isomorphic to
\(\Integers \times \Integers\). Suppose \(a/b, c/d \in \Rationals\) generate the
corresponding subgroups isomorphic to
\(\Integers \times 0\Integers\) and \(0\Integers \times \Integers\)
respectively, so
\(\Integers \times \Integers \isom \cycsgp{a/b} \times \cycsgp{c/d}\).

Then
\(bd[\cycsgp{a/b} \times \cycsgp{c/d}]
  = \set{mad + nbc : m, n \in \Integers}\), as
\(\Rationals\) is abelian. But by B\'ezout's Theorem, \(mad + nbc\) can take on
precisely the values in \([\gcd(ad, bc)]\Integers\), so in fact
\begin{equation*}
 \cycsgp{a/b} \times \cycsgp{c/d}
 \isom \bracks[\bigg]{\frac{\gcd(ad, bc)}{bd}}\Integers
 \isom \Integers
\end{equation*}
giving \(\Integers \times \Integers \isom \Integers\), which is a contradiction,
as previously established.

So also \(\Rationals\) is not decomposable.

\section{\((\Reals, +)\)}

Let \(\mathcal H\) be a Hamel basis for \(\Reals\) as a vector space over
\(\Rationals\). Then any partition of \(\mathcal H\) as a disjoint union
\(\mathcal H = \mathcal H_1 \cup \mathcal H_2\) gives rise to two subgroups
\(G_1 = \vspan \mathcal H_1, G_2 = \vspan \mathcal H_2 \sgp \Reals\). These are
subgroups since they are subspaces by definition.

Now we can apply the Direct Product Theorem, since
\begin{enumerate}[label=(\roman*)]
 \item
  By linear independence, if \(x_1 \in G_1, x_2 \in G_2\) with \(x_1 = x_2\),
  then \(x_1 = x_2 = 0\), so \(G_1 \cap G_2\) is precisely \(\set{0}\).
 \item
  \(\Reals\) is abelian, so trivially,
  \(x_1 + x_2 = x_2 + x_2 \Forall x_1 \in G_1, x_2 \in G_2\).
 \item
  Each \(x \in \Reals\) can be written as a linear combination of elements of
  \(\mathcal H\), by definition. Then simply splitting this linear combination
  into linear combinations of the elements in \(\mathcal H_1\) and
  \(\mathcal H_2\), we get \(x_1 \in G_1, x_2 \in G_2\) with \(x_1 + x_2 = x\).
\end{enumerate}
So \(\Reals \isom G_1 \times G_2\) for any such choice.

To be explicit, a possible choice here might be
\(\mathcal H_1 = \mathcal H \cap \Rationals\),
\(\mathcal H_2 = \mathcal H \setminus \Rationals\), or possibly replacing
\(\Rationals\) with some more imaginative subsets of \(\Reals\). Here, we know
\(\abs{\mathcal H_1} = 1\) precisely, as there must be a rational in
\(\mathcal H\), so neither of \(G_1, G_2\) is trivial.

\section{\((\Complex, +)\)}

\(\Complex\) is quite straightforwardly isomorphic to \(\Reals \times \Reals\)
via \(z \mapsto (\Re z, \Im z)\).
%ImRe??

\section{\((\Rationals^\times, \cdot)\)}

This is isomorphic to \(C_2 \times (\Rationals^+, \cdot)\) via
\(x \mapsto (\sgn x, \abs x)\) (where \(\sgn x \defeq x/\abs x\)).

\section{\((\Reals^\times, \cdot)\)}

This is isomorphic to \(C_2 \times (\Reals^+, \cdot)\) via
\(x \mapsto (\sgn x, \abs x)\).

\section{\((\Complex^\times, \cdot)\)}

This is isomorphic to \(S^1 \times (\Reals^+, \cdot)\) via
\(re^{i\theta} \mapsto (e^{i\theta}, r)\).

\section{\(\parens[\big]{\set{a / 2^n : a, n \in \Integers}, +}\)}

\section{\((\Rationals^+, \cdot)\)}

\section{\((\Reals^+, \cdot)\)}

This is isomorphic to \((\Reals, +)\) via \(x \mapsto \log x\), so we can reuse
the previous construction.

\section{\(S^1 \isom \Reals / \Integers\)}

\section{\(\Rationals / \Integers\)}

\end{document}
