% Compiling with
% latexmk -halt-on-error -shell-escape -synctex=1 -pdf
% (Recommend using a latexmkrc file so as just to run latexmk -pvc, for example)
% Probably you can achieve the same with an inordinate number of invocations of
% pdflatex -halt-on-error -shell-escape -synctex=1

% fleqn aligns equations to the left, a4 paper size, 11pt font, article class
\documentclass[fleqn,a4paper,11pt]{article}
\title{On Decompositions of Infinite Groups}
\author{A. Horn, D. Simms, et al.}

\usepackage{mymaths}
\usepackage{mystyle}

\begin{document}
\maketitle

A \emph{decomposition} of a group \(G\) is a pair of groups \(A, B\) such that
neither of \(A, B\) is trivial, and \(G \isom A \dprod B\).

\section{\((\Z, +)\)}

\(\Z\) is not decomposable. Suppose we had \(\Z \isom G \dprod H\) for
nontrivial \(G\), \(H\). Then since the subgroups of \(\Z\) are of the form
\(k\Z\), we have \(G \isom n\Z\) and \(H \isom m\Z\) for some \(n, m\).

Since \(k\Z \isom \Z\), this implies \(\Z \isom G \dprod H \isom \Z \dprod \Z\).
This is not possible since \(\Z\) has \(2\Z\) as its only subgroup of index two,
but \(\Z \dprod \Z\) has two subgroups of index two: \(2\Z \dprod \Z\) and \(\Z
\dprod 2\Z\).

Hence \(\Z\) is not decomposable.

\section{\((\Q, +)\)}

Suppose we had \(\Q \isom G \dprod H\) for some nontrivial groups \(G, H\). Both
\(G\) and \(H\) have a subgroup isomorphic to \(\Z\), since each contains a
nonzero element, and each element of \(\Q\) has infinite order.

Hence \(\Q\) has a subgroup isomorphic to \(\Z \dprod \Z\). Suppose
\(a/b, c/d \in \Q\) generate the corresponding subgroups isomorphic to
\(\Z \dprod 0\Z\) and \(0\Z \dprod \Z\) respectively, so
\(\Z \dprod \Z \isom \cycsgp{a/b} \dprod \cycsgp{c/d}\). Without loss of
generality, both generators are positive.

Then, however, \(bc(a/b) = ad(c/d)\). This implies that in the subgroup
isomorphic to \(\Z \dprod \Z\), we have \((bc, 0) = (0, ad)\), which is a
contradiction as neither of these terms can be \(0\) if \(a/b, c/d\) are to be
generators.

So also \(\mathbb Q\) is not decomposable.

\section{\((\R, +)\)}

Let \(\mathcal H\) be a Hamel basis for \(\R\) as a vector space over \(\Q\).
Then any partition of \(\mathcal H\) as a disjoint union
\(\mathcal H = \mathcal H_1 \cup \mathcal H_2\) gives rise to two subgroups
\(G_1 = \vspan \mathcal H_1, G_2 = \vspan \mathcal H_2 \sgp \R\). These are
subgroups since they are subspaces by definition.

Now we can apply the Direct Product Theorem, since
\begin{enumerate}[label=(\roman*)]
 \item
  By linear independence, if \(x_1 \in G_1, x_2 \in G_2\) with \(x_1 = x_2\),
  then \(x_1 = x_2 = 0\), so \(G_1 \cap G_2\) is precisely \(\set{0}\).
 \item
  \(\R\) is abelian, so trivially,
  \(x_1 + x_2 = x_2 + x_2 \Forall x_1 \in G_1, x_2 \in G_2\).
 \item
  Each \(x \in \R\) can be written as a linear combination of elements of
  \(\mathcal H\), by definition. Then simply splitting this linear combination
  into linear combinations of the elements in \(\mathcal H_1\) and
  \(\mathcal H_2\), we get \(x_1 \in G_1, x_2 \in G_2\) with \(x_1 + x_2 = x\).
\end{enumerate}
So \(\R \isom G_1 \dprod G_2\) for any such choice. Then, so long as neither of
\(\mathcal H_1, \mathcal H_2\) were empty, \(G_1\) and \(G_2\) will be
nontrivial, giving a decomposition of \(\R\).

Note that really this implies that anything that is the additive group of a
vector space over \emph{some} field, of dimension at least two, is decomposable
(with a generous sprinkling of the axiom of choice).

\section{\((\CC, +)\)}

\(\CC\) is quite straightforwardly isomorphic to \(\R \dprod \R\) via \(z
\mapsto (\Re z, \Im z)\).
%ImRe??

\section{\((\Q^\times, \cdot)\)}

This is isomorphic to \(C_2 \dprod (\Q^+, \cdot)\) via
\(x \mapsto (\sgn x, \abs x)\) (where \(\sgn x \defeq x/\abs x\)).

\section{\((\R^\times, \cdot)\)}

This is isomorphic to \(C_2 \dprod (\R^+, \cdot)\) via
\(x \mapsto (\sgn x, \abs x)\).

\section{\((\CC^\times, \cdot)\)}

This is isomorphic to \(S^1 \dprod (\R^+, \cdot)\) via
\(re^{i\theta} \mapsto (e^{i\theta}, r)\).

\section{\((\Q^+, \cdot)\)}

Consider the following two subgroups:
\begin{align*}
 G &= \set{2^n : n \in \Z} \isom \Z \\
 H &= \set[\Big]{\frac ab : a, b \in \Z, 2 \ndivides a, b}
\end{align*}
The fact that they are subgroups follows quite quickly from the efficient
subgroup criterion.

Now we can apply the Direct Product Theorem to these, since
\begin{enumerate}[label=(\roman*)]
 \item
  Let \(2^n\) be an element of \(G\), and \(a/b\) be an element of \(H\).
  Suppose \(2^n\) is an integer.

  If \(a/b = 2^n\), then \(a = 2^n b\), so \(2^n \divides a\). Hence \(2^n = 1\).

  Suppose instead \(2^n\) is fractional. Then, we get \(2^{-n} a = b\), so
  \(2^{-n} \divides b\), which is not possible.

  So \(G \cap H = {1}\).
 \item
  All elements of \(G\) and \(H\) commute trivially since \(\Q^+\) is abelian.
 \item
  If \(p/q \in \Q\) is a general element with \(p, q\) in lowest terms, then
  either only \(p\) has a largest factor of \(2^n\) for some \(n\), so
  \(p/q = 2^n \cdot (p/2^n)/q\), or \(q\) has a largest factor of \(2^n\) for
  some \(n\), so \(p/q = 2^{-n} \cdot p/(q/2^n)\), or neither does, and
  \(p/q = 1 \cdot p/q\).

  In each case, we have written \(p/q = gh\) for some \(g \in G, h \in H\).
\end{enumerate}
So \((\Q^+, \cdot) \isom G \dprod H \isom \Z \dprod H\) and \(\Q\) is
decomposable. This process can be repeated with the next prime number (\(3\),
probably) to decompose \(H\) as \(\Z \dprod H'\), and so on.

\section{\((\R^+, \cdot)\)}

This is isomorphic to \((\R, +)\) via \(x \mapsto \log x\), so we can reuse the
previous construction.

\section{\(S^1 \isom \R / \Z\)}

\section{\(\Q / \Z\)}

Let
\begin{align*}
 G &= \set{a/2^n : a \in \Z, n \in \Z_{\ge 0}} \\
 H &= \set{a/m : a \in \Z, m \in \Z_{\ge 0}, 2 \ndivides m}
\end{align*}
These are subgroups of \(\Q\) containing \(\Z\), so
\(G / \Z, H / \Z \sgp \Q / \Z\).

We can apply the Direct Product Theorem, since
\begin{enumerate}[label=(\roman*)]
 \item
  \(G \cap H\) is precisely the integers, since the only power of two with no
  factor of two is \(1\). But the quotient map \(\Q \to \Q/\Z\) precisely takes
  all the integers to the identity coset, so \(G/\Z \cap H/\Z\) is trivial.
 \item
  \(\Q/\Z\) is abelian, so trivially, all elements of \(G/\Z\) and \(H/\Z\)
  commute.
 \item
  In \(\Q\), let \(b/(2^n m)\) be a general element, with \(2 \ndivides m\). Now
  by B\'ezout's Theorem, the diophantine equation  \(mx + 2^n y = 1\) has a
  solution for \(x, y \in \Z\), since \(\gcd(2^n, m) = 1\). But this precisely
  means that \(1/(2^n m) = x/2^n + y/m\), a sum of elements of \(G, H\).

  Then by well-definedness of the quotient map, we immediately get that any
  given coset \(b/(2^n m) + \Z \in \Q / \Z\) is just
  \((x/2^n + \Z) + (y/m + \Z)\).
\end{enumerate}
Hence \(\Q/\Z \isom G/\Z \dprod H/\Z\), which is a nontrivial decomposition.

\section{\(\parens[\big]{\set{a / 2^n : a \in \Z, n \in \Z_{\ge 0}}, +}
           \isom \bigcup\limits_{n \in \N} 2^{-n} \Z \)}

\section{\(\parens[\big]{
            \set{a / m : a \in \Z, n \in \Z_{\ge 0}, 2 \ndivides m}, +}\)}

\section{\(\parens[\big]{\set{a / 2^n : a \in \Z, n \in \Z_{\ge 0}}, +} / \Z\)}

Call this group \(G\) for convenience. Also we refer to elements of \(G\) as
their unique coset representatives in \(\intco{0, 1}\). Any infinite subgroup
\(H \sgp G\) must be \(G\), since in order to be infinite, \(H\) must contain
elements that have arbitrary large denominators written in lowest terms, since
there are finitely many (\(2^n\)) elements that can be written with any
denominator \(2^n\).

But if we have \(x/2^n\) where \(2 \ndivides x\), then
\(xy \mcong 1 \pmod{2^n}\) is solvable since \(x, 2^n\) are coprime. So
\(\cycsgp{x/2^n} \sgp H\) contains \(1/2^n\) and hence also \(a/2^n\) for all
\(a\), which gives all elements of denominator at most \(2^n\). Since the
denominators found in \(H\) are arbitrarily large, \(H\) contains all elements,
and \(H = G\).

But if \(G \isom A \dprod B\) nontrivially, then \(A, B\) are both isomorphic to
proper subgroups of \(G\) (since there is a non-identity \(b \in B\) implying
\((e, b) \notin A \times \set e\) and vice versa), so \(A, B\) are both finite.
But then \(A \dprod B\) is finite with order \(\abs A \abs B\), which
contradicts the fact that \(G\) is infinite (eg since \(G\) contains \(1/2^n\)
for all \(n \in \Z_{\ge 0}\)).

So \(G\) is not decomposable.

This argument works for the family of similar groups given by fractions with
denominators power of any prime, mod \(\Z\).

\section{\(\R / \Q\)}

Tentatively, this is a quotient space of \(\R\) over \(\Q\), so by
the same argument as for \(\R\), is decomposable.

\(\Q\) is a subspace of \(\R\) over \(\Q\) as it is just the span of any
(nonzero) rational number (eg \(1\)). Particularly, once \(\R\) is equipped with
a Hamel basis \(\mathcal H\), \(\Q\) is the kernel of the vectors space
homomorphism \(\R \to \R\) given by ``ignore the rational term in the linear
combination of basis vectors'', which has image isomorphic to \(\R / \Q\).

\end{document}
